\documentclass[a4paper]{article}
\usepackage[utf8]{inputenc}
\usepackage[french]{babel}
\usepackage[T1]{fontenc}
\usepackage{geometry}
\geometry{hmargin=1cm,vmargin=1cm}
\usepackage{amsmath}
\usepackage{amssymb}
\usepackage{fp}
\usepackage{tikz}
\usepackage{tabularx}

\usepackage{array,multirow}

\pagestyle{empty}

\setlength\parindent{0pt}
\newcolumntype{Y}{>{\centering\arraybackslash}X}
\setlength\extrarowheight{2pt} % fix cell height


\begin{document}

%%%%%%%% DEFINITION DU NUMERO DE SUJET %%%%%%%%
%                                             %
                \def\Sujet{2}                 %
%                                             %
%%%%%%%%%%%%%%%%%%%%%%%%%%%%%%%%%%%%%%%%%%%%%%%


\FPeval\SujetPrec{trunc(\Sujet-1,0)}

\newcommand\SetSeed[2]{%
    \FPeval\Seed{trunc(#1*431*#2*79,0)}%
    \FPseed=\Seed%
}

% \R{x}{y}{z} génère une variable entière aléatoire \Rx compris entre y et z
\newcommand\R[3]{\expandafter\FPeval\csname R#1\endcsname{trunc(#2+random*(#3-#2),0)}}
% \SR{x}{y}{z} génère une variable entière aléatoire \Rx compris entre y et z ou entre -z et -y
\newcommand\SR[3]{\expandafter\FPeval\csname R#1\endcsname{trunc(trunc(round(random,0)*2-1,0)*trunc(#2+random*(#3-#2),0),0)}}
% \S{x} affiche x précédé d'un + si positif
\newcommand\Plus[1]{\FPifgt#1{0}+#1\else#1\fi}

% \NE{x}{y} égal 1 si x = y et 0 sinon
\newcommand\NE[2]{trunc(1-abs((#1)-(#2))/1000,0)}
% \EZ{x}{y}{z} affiche y si x = 0 et et z sinon
\newcommand\EZ[3]{\FPifeq#1{0}{#2}\else#3\fi}

% (ax+b)(cx+d)
\newcommand\ExoDoubleDistrib{%
    \SR{a}{1}{10}%
    \SR{b}{1}{10}%
    \SR{c}{1}{10}%
    \SR{d}{1}{10}%
    $(\Ra x \Plus{\Rb})(\Rc x \Plus{\Rd})$%
}
\newcommand\SolDoubleDistrib{%
    \SR{a}{1}{10}%
    \SR{b}{1}{10} %
    \SR{c}{1}{10}%
    \SR{d}{1}{10}%
    \FPeval\Va{trunc((\Ra)*(\Rc),0)}%
    \FPeval\Vb{trunc((\Ra)*(\Rd) + (\Rb)*(\Rc),0)}%
    \FPeval\Vc{trunc((\Rb)*(\Rd),0)}%
    $\Va x^2 \Plus{\Vb} x \Plus{\Vc}$%
}

% Variante du précédent : (ax+b)(c+dx)
\newcommand\ExoDoubleDistribA{%
    \SR{a}{1}{10}%
    \SR{b}{1}{10}%
    \SR{c}{1}{10}%
    \SR{d}{1}{10}%
    $(\Ra x \Plus{\Rb})(\Rc \Plus{\Rd}x)$%
}
\newcommand\SolDoubleDistribA{%
    \SR{a}{1}{10}%
    \SR{b}{1}{10} %
    \SR{c}{1}{10}%
    \SR{d}{1}{10}%
    \FPeval\Va{trunc((\Ra)*(\Rd),0)}%
    \FPeval\Vb{trunc((\Ra)*(\Rc) + (\Rb)*(\Rd),0)}%
    \FPeval\Vc{trunc((\Rb)*(\Rc),0)}%
    $\Va x^2 \Plus{\Vb} x \Plus{\Vc}$%
}

% Variante : (ax²+b)(cx+d)
\newcommand\ExoDoubleDistribB{%
    \SR{a}{1}{10}%
    \SR{b}{1}{10}%
    \SR{c}{1}{10}%
    \SR{d}{1}{10}%
    $(\Ra x^2 \Plus{\Rb})(\Rc x \Plus{\Rd})$%
}
\newcommand\SolDoubleDistribB{%
    \SR{a}{1}{10}%
    \SR{b}{1}{10} %
    \SR{c}{1}{10}%
    \SR{d}{1}{10}%
    \FPeval\Va{trunc((\Ra)*(\Rc),0)}%
    \FPeval\Vb{trunc((\Ra)*(\Rd) + (\Rb)*(\Rc),0)}%
    \FPeval\Vc{trunc((\Rb)*(\Rd),0)}%
    $\Va x^3 \Plus{\Vb} x^2 \Plus{\Vc}$%
}

% (ax ± b)^2
\newcommand\ExoIRa{%
    \SR{a}{1}{10}%
    \SR{b}{1}{10}%
    $(\Ra x \Plus{\Rb})^2$%
}
\newcommand\SolIRa{%
    \SR{a}{1}{10}%
    \SR{b}{1}{10}%
    \FPeval\Va{trunc((\Ra)*(\Ra),0)}%
    \FPeval\Vb{trunc(2*(\Ra)*(\Rb),0)}%
    \FPeval\Vc{trunc((\Rb)*(\Rb),0)}%
    $\Va x^2 \Plus{\Vb} x \Plus{\Vc}$%
}

% Variante : (a ± bx)^2
\newcommand\ExoIRaA{%
    \SR{a}{1}{10}%
    \SR{b}{1}{10}%
    $(\Ra \Plus{\Rb}x)^2$%
}
\newcommand\SolIRaA{%
    \SR{a}{1}{10}%
    \SR{b}{1}{10}%
    \FPeval\Va{trunc((\Ra)*(\Ra),0)}%
    \FPeval\Vb{trunc(2*(\Ra)*(\Rb),0)}%
    \FPeval\Vc{trunc((\Rb)*(\Rb),0)}%
    $\Vc x^2 \Plus{\Vb} x \Plus{\Va}$%
}

% (ax + b)(ax - b)
\newcommand\ExoIRb{%
    \SR{a}{1}{10}%
    \R{b}{1}{10}%
    $(\Ra x + \Rb)(\Ra x - \Rb)$%
}
\newcommand\SolIRb{%
    \SR{a}{1}{10}%
    \R{b}{1}{10}%
    \FPeval\Va{trunc((\Ra)*(\Ra),0)}%
    \FPeval\Vb{trunc((\Rb)*(\Rb),0)}%
    $\Va x^2 - \Vb$%
}

% (a + bx)(a - bx)
\newcommand\ExoIRbB{%
    \SR{a}{1}{10}%
    \R{b}{1}{10}%
    $(\Ra + \Rbx)(\Ra - \Rb x)$%
}
\newcommand\SolIRbB{%
    \SR{a}{1}{10}%
    \R{b}{1}{10}%
    \FPeval\Va{trunc((\Ra)*(\Ra),0)}%
    \FPeval\Vb{trunc((\Rb)*(\Rb),0)}%
    $\Vb x^2 - \Va$%
}

% Equation ax + b = c
\newcommand\ExoEquationA{%
    % ax + c = ab + c
    %           Va
    \SR{a}{2}{10}%
    \SR{b}{1}{10}%
    \SR{c}{11}{100}%
    \FPeval\Va{trunc((\Ra)*(\Rb)+(\Rc),0)}%
    $\Ra x \Plus{\Rc} = \Va$%
}
\newcommand\SolEquationA{%
    \SR{a}{2}{10}%
    \SR{b}{1}{10}%
    \SR{c}{11}{100}%
    $x = \Rb$%
}

% Variante ax + b = cx + d
\newcommand\ExoEquationB{%
    % (a + d)x + c = dx + (ab + c)
    %   Vb                   Va
    \SR{a}{2}{10}%
    \SR{b}{1}{10}%
    \SR{c}{11}{100}%
    \SR{d}{2}{10}%
    \FPeval\Va{trunc((\Ra)*(\Rb)+(\Rc),0)}%
    \FPeval\Vb{trunc((\Ra)+(\Rd),0)}%
    $\EZ{\Vb}{}{\Vb x} \Plus{\Rc} = \Rd x \Plus{\Va}$%
}
\newcommand\SolEquationB{%
    \SR{a}{2}{10}%
    \SR{b}{1}{10}%
    \SR{c}{11}{100}%
    \SR{d}{2}{10}%
    $x = \Rb$%
} % Charge les modèles d'exos

\input{\SujetPrec} % Charge la \Sol du sujet précédent
\input{\Sujet} % Charge l'\Exo du sujet actuel (\Exo est redéfinie mais \Sol ne l'est pas)

\foreach \n in{1,...,10}  {

\begin{tabularx}{\linewidth}{lYr}
Sujet \no\Sujet & \textbf{Entraînement au calcul} & Élève \no\n
\end{tabularx}

\medskip

\SetSeed{\Sujet}{\n}

\Sol

\medskip

\SetSeed{\Sujet}{\n}


\Exo

\ifodd\n\vfill\else\newpage\fi

}


\end{document}